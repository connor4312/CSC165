\documentclass[boldsans]{article}
\usepackage{amsmath,amssymb}
\usepackage{hyperref}
\title{CSC165 Fall 2014, Assignment 2}
\author{Anish Krishna, Connor Peet, George Wu}
\renewcommand{\today}{~}
\hypersetup{pdfpagemode=Fullscreen,
  colorlinks=true,
  linkfileprefix={}}
\newcommand{\floor}[1]{\lfloor #1\rfloor}
\begin{document}
\maketitle
\noindent
The aim of this assignment is for you to practice devising and
presenting proofs.  You may work in groups of no more than three
students, and you should produce a single solution in a PDF file named
\texttt{a2.pdf}, submitted to
\href{https://markus.cdf.toronto.edu/csc165-2014-09}{MarkUs}.

You will receive 20\% of the marks for any question you either leave
blank, or write ``I cannot answer this.''  You will receive 0 for any
false claim you ``prove,'' or any true claim you ``disprove.''

\begin{enumerate}
\item Proving the statement 1.1 true.
    \begin{description}
    \item Assume $x \in \mathbb{R}$ \# generic
        \begin{description}
        \item Assume $y \in \mathbb{R}$ \# generic
            \begin{description}
            \item Assume $x > y$ \# Antecedent
                \begin{description}
                \item Let $d = x - y$
                \item Then $y - 1 < x - 1$
                \item Then $y - 1 < \floor{y} \leq y$ and $x - 1 < \floor{x} \leq x$ \# By defintion, there cannot be an integer greater than $\floor{y}$ which is also greater than $y$ itself. Therefore, the $\floor{y}$ must be at last 1 less than y, and at most equal to y (definition: $\floor{y} \leq y$).
                \item Then there is one possible value for $\floor{x}$ and $\floor{y}$ \# Floors are integers, by definition.
                \item Then $y + d - 1 < \floor{x} \leq y + d$
                \item Then $y - 1 < y + d - 1 < \floor{y} \leq \floor{x} \leq y \leq y + d$ \# Combine the inequalities - we can do this because we know that the floors both return a single integer.
                \item Then $\floor{y} \leq \floor{x}$
                \end{description}
            \item Conclude $x > y \implies \floor{x} \geq \floor{y}$
            \end{description}
        \item Conclude $\forall y \in \mathbb{R}, x < y \implies \floor{x} \geq \floor{y}$
        \end{description}
    \item Conclude $\forall x \in \mathbb{R}, \forall y \in \mathbb{R}, x < y \implies \floor{x} \geq \floor{y}$
    \end{description}
\item Proving the statment 1.2 true.
    \begin{description}
    \item Assume $x \in \mathbb{R}$ \# generic
    \item Assume $e \in \mathbb{R}^+$ \# generic, indent remaining for formatting purposes
        \begin{description}
        \item Let d = e + 2
        \item Then $d \in \mathbb{R}^+$ \# $\mathbb{R}^+$ closed by addition
        \item Assume $w \in \mathbb{R}$ \# generic
            \begin{description}
            \item Assume $w \in \mathbb{R}$ \# generic
                \begin{description}
                \item Assume $\mid x - w \mid < d$ is true \# antecedent
                    \begin{description}
                    \item Then $y - 1 < \floor{y} \leq y$ and $x - 1 < \floor{x} \leq x$ \# See proof 1.1
                    \item Then $\floor{x} = x \pm 1$ and $\floor{y} = y \pm 1$ \# the floor of a value will therefore always be within 1 of the value
                    \item Then $\mid x - q \mid < e + 2$ \# Antecedent
                    \item Then $\mid (\floor{x} \pm 1) - (\floor{y} \pm 1) \mid < e + 2$
                    \item Then $\mid \floor{x} - \floor{y} \mid + 2 < e$ \# move the $\pm$s out of the absolute value; they become simply positive two.
                    \item Then $\mid \floor{x} - \floor{y} \mid < e$
                    \end{description}
                        
                \item Then $\mid x - w \mid < d \implies \mid \floor{x} - \floor{y} \mid < e$
                \end{description}
            \item Then $\forall w \in \mathbb{R}, \mid x - w \mid < d \implies \mid \floor{x} - \floor{y} \mid < e$
            \end{description}
        \item Then $\exists d \in \mathbb{R}^+, \forall w \in \mathbb{R}, \mid x - w \mid < d \implies \mid \floor{x} - \floor{y} \mid < e$
        \end{description}
    \item Then $\forall e \in \mathbb{R}^2, \exists d \in \mathbb{R}^+, \forall w \in \mathbb{R}, \mid x - w \mid < d \implies \mid \floor{x} - \floor{y} \mid < e$
    \item Then $\forall x \in \mathbb{R}, \forall e \in \mathbb{R}^2, \exists d \in \mathbb{R}^+, \forall w \in \mathbb{R}, \mid x - w \mid < d \implies \mid \floor{x} - \floor{y} \mid < e$
    \end{description}

\item For $x\in\mathbb{R}$, define  $\floor{x}$ by:
    \begin{equation*}
        \floor{x} \in\mathbb{Z} \wedge \floor{x}
        \leq x \wedge (\forall z \in\mathbb{Z}, z \leq x \Rightarrow z
        \leq \floor{x}).
    \end{equation*}
    \ldots where $\mathbb{Z}$ stands for the set of integers, and
    $\mathbb{R}$ stands for the set of real numbers.
    Use the definition of $\floor{x}$ to prove or disprove each of the
    following claims, using the structured proof technique from this
    course.  \textbf{Note:} You must use the definition given here,
    not some other (possibly equivalent) definition for $\floor{x}$.
  \begin{description}
  \item[Claim 1.1:] ~ 
    \begin{equation*}
      \forall x \in\mathbb{R}, \forall y \in\mathbb{R}, x > y \Rightarrow
      \floor{x} \geq \floor{y}.
    \end{equation*}
  \item[Claim 1.2:] ~
    \begin{equation*}
  \forall x\in\mathbb{R}, \forall e \in\mathbb{R}^+, \exists
  d \in\mathbb{R}^+, \forall w\in\mathbb{R}, |x-w|< d
  \Rightarrow |\floor{x}  - \floor{w}| < e
    \end{equation*}
  \item[Claim 1.3:] ~
    \begin{equation*}
        \exists x\in\mathbb{R}, \forall e \in\mathbb{R}^+, \exists
  d \in\mathbb{R}^+, \forall w\in\mathbb{R}, |x-w|< d
  \Rightarrow |\lfloor x\rfloor - \lfloor w\rfloor| < e
    \end{equation*}
  \item[Claim 1.4:] ~
    \begin{equation*}
      \exists x \in\mathbb{R}, \floor{x + 1} \neq \floor{x} + 1
    \end{equation*}
  \end{description}

\item \textbf{Prove} or \textbf{disprove} the claim, and
  \textbf{prove} or \textbf{disprove} the converse:
  \begin{description}
  \item[Claim 2.1:] ~
    \begin{equation*}
      \forall n \in\mathbb{N}, \left(\exists k\in\mathbb{N}, n =
        5k+2\right) \Rightarrow \left(\exists j\in\mathbb{N}, n^2 = 5j+4\right)
    \end{equation*}
  \item[Claim 2.2:] ~
    \begin{equation*}
      \forall m, n\in\mathbb{N}, \left(\exists k\in\mathbb{N}, m =
        7k+3\right) \wedge \left(\exists j\in\mathbb{N}, n = 7j + 4\right)
      \Rightarrow \left(\exists i\in\mathbb{N}, mn = 7i + 5\right)
    \end{equation*}
  \end{description}

\end{enumerate}

\end{document}

% LocalWords:  brachial