\documentclass{article}
\usepackage{amsmath,amssymb}
\usepackage{hyperref}
\title{CSC165 Fall 2014, Assignment 2}
\author{Anish Krishna, Connor Peet, George Wu}
\renewcommand{\today}{~}
\hypersetup{pdfpagemode=Fullscreen,
  colorlinks=true,
  linkfileprefix={}}
\newcommand{\floor}[1]{\lfloor #1\rfloor}
\begin{document}
\maketitle
\noindent
The aim of this assignment is for you to practice devising and
presenting proofs.  You may work in groups of no more than three
students, and you should produce a single solution in a PDF file named
\texttt{a2.pdf}, submitted to
\href{https://markus.cdf.toronto.edu/csc165-2014-09}{MarkUs}.

You will receive 20\% of the marks for any question you either leave
blank, or write ``I cannot answer this.''  You will receive 0 for any
false claim you ``prove,'' or any true claim you ``disprove.''

\begin{enumerate}
\item Proving the statement 1.1 true.
    \begin{description}
    \item Assume $x \in \mathbb{R}$ \# generic
        \begin{description}
        \item Assume $y \in \mathbb{R}$ \# generic
            \begin{description}
            \item Assume $x > y$ \# Antecedent
                \begin{description}
                \item Then $\forall z \in \mathbb{Z}, z \leq x \implies z \leq \floor{x}$ and $\forall z \in \mathbb{Z}, z \leq y \implies z \leq \floor{y}$\# Definition of floor
                \item Then $\forall z \in \mathbb{Z}, z \leq y \implies (z \leq \floor{x} \land z \leq \floor{y})$ \# $x < y$, so by the antecedent $z \leq y \implies z \leq x$
                \item Let there be $d = (x - y) / 1$ \# Treat as integer division.
                \item Then $d \in \mathbb{N}_0$ \# d must be $\geq$ zero because $x > y$, and closed by division.
                \item Then $\forall z \in \mathbb{Z}, z \leq y \implies (z + d \leq \floor{x} \land z \leq \floor{y})$ \# Make the $x$ part of the and statement equivalent to its definition.
                \item Then $\floor{x} \leq \floor{y}$ \# $z < z + d$
                \end{description}
            \item Conclude $x > y \implies \floor{x} \geq \floor{y}$
            \end{description}
        \item Conclude $\forall y \in \mathbb{R}, x < y \implies \floor{x} \geq \floor{y}$
        \end{description}
    \item Conclude $\forall x \in \mathbb{R}, \forall y \in \mathbb{R}, x < y \implies \floor{x} \geq \floor{y}$
    \end{description}
\item Proving the statment 1.2 true.
    \begin{description}
    \item Assume $x \in \mathbb{R}, e \in \mathbb{R}^+$ \# generic
        \begin{description}
        \item Let d = e + 2
        \item Then $d \in \mathbb{R}^+$ \# $\mathbb{R}^+$ closed by addition
        \item Assume $w \in \mathbb{R}$ \# generic
            \begin{description}
            \item Assume $\mid x - w \mid < d$ is true \# antecedent
                \begin{description}
                \item Then $w - 1 < \floor{w} \leq y$ and $x - 1 < \floor{x} \leq x$ \# By definition of floor, $\forall z \in \mathbb{Z}, z \leq x \implies z \leq \floor{x}$ and $\floor{x} \leq x$
                \item Then $\floor{x} = x \pm 1$ and $\floor{w} = w \pm 1$ \# therefore the floor of a value will therefore always be within $\pm 1$ of the value
                \item Then $\mid (\floor{x} \pm 1) - (\floor{w} \pm 1) \mid < e + 2$ \# Using the antecedent, sub in for $x$ and $w$.
                \item Then $\mid \floor{x} - \floor{w} \pm 2 \mid < e + 2$ \# Associative property of addition
                \item Then $\mid \floor{x} - \floor{w} \mid + 2 < e + 2$ \# move the $\pm$s out of the absolute value; they become simply positive two at maximum and zero at minimum, so the inequality holds.
                \item Then $\mid \floor{x} - \floor{w} \mid < e$ \# Subtract 2 from both sides.
                \end{description}

            \item Then $\mid x - w \mid < d \implies \mid \floor{x} - \floor{y} \mid < e$
            \end{description}
        \item Then $\exists d \in \mathbb{R}^+, \forall w \in \mathbb{R}, \mid x - w \mid < d \implies \mid \floor{x} - \floor{y} \mid < e$
        \end{description}
    \item Then $\forall x \in \mathbb{R}, \forall e \in \mathbb{R}^2, \exists d \in \mathbb{R}^+, \forall w \in \mathbb{R}, \mid x - w \mid < d \implies \mid \floor{x} - \floor{y} \mid < e$
    \end{description}
\item Proving the statement 1.3 true.
    \begin{description}
    \item Let $x = 0.5$ \# 0.5 is a positive real number
    \item Then $x \in \mathbb{R}$
    \item Assume $e \in \mathbb{R}^+$ \# Generic
        \begin{description}
        \item Let $d = 0.1$
        \item Then $d \in \mathbb{R}^+$ \# 0.1 is a positive real number
        \item Assume $w \in \mathbb{R}$ and that $\mid x - w \mid < d$
            \begin{description}
            \item Then $0.4 < w < 0.6$ \# Difference between x (0.5) and w is less than 0.1.
            \item Then $\floor{w} = \floor{x} = 0$ \# From definition of floor
            \item Then $\mid \floor{x} - \floor{w} \mid = 0$ \# Zero minus zero is simply zero.
            \item Then $\mid \floor{x} - \floor{w} \mid < e$ \# Zero is less than all positive real numbers, by definition of positive.
            \end{description}
        \item Then $\forall w\in\mathbb{R}, |x-w|< d \implies |\lfloor x\rfloor - \lfloor w\rfloor| < e$
        \end{description}
    \item Then $\exists x\in\mathbb{R}, \forall e \in\mathbb{R}^+, \exists d \in\mathbb{R}^+, \forall w\in\mathbb{R}, |x-w|< d \implies |\lfloor x\rfloor - \lfloor w\rfloor| < e$
    \end{description}
\item Proving 1.4 False.
		\begin{description}
		\item The negation of claim 1.4 would be:
		\item $\forall x \in \floor{x+1} = \floor{x} + 1$
				\begin{description}
				\item Assume $x \in \mathbb{R}$ \# Generic
						\begin{description}
						\item $\exists Z_{0}, Z_1$ such that $Z_0 \leq x+1$, $Z_1 \leq x$
						\item Then $Z_0 \in \mathbb{Z}$
						\item Then, $Z_{0} - 1 \leq x$ \# This is the largest integer less than x, since subtracting one from the largest integer less than or equal to x+1 will give the largest integer less than x.
						\item But, $Z_1$, by definition is also the largest integer less than or equal to x.
						\item Then $Z_0 = Z_1 + 1$
						\item Substituting for the values of $Z_0$, and $Z_1$, $\floor{x+1} = \floor{x}+1$
						\end{description}
				\item Then, $\floor{x+1} = \floor{x}+1$
				\end{description}
		\item Then $\forall x \in \floor{x+1} = \floor{x} + 1$
		\end{description}
\item Proving 2.1 is true, and proving the converse of 2.1 is false.
		\begin {description}
				\item Let n be any natural number. \# Domain assumption.
				\begin{description}
						\item Let $k_0$ be a natural number. \# Thus, $k \in \mathbb{N}$
						\item Then, $n = 5k_0 + 2$ \#Antecedent assumption
						\item $n^2 = (5k_0 + 2)(5k_0 + 2) = 25 k_0^2 + 20k_0 + 4 = 5(5k_0^2 + 4k_0) + 4$ \#Simplifying $n^2$
						\item Then $5k_0^2 + 4k_0 \in \mathbb{N}$ \# natural numbers closed under addition, multiplication, and exponentation.
						\item Then $j_0 = 5k_0^2 + 4k_0$
						\item Then $\exists j \in \mathbb{N}$ such that $n^2 = 5j+4$
				\end {description}
				\item Then $\forall n \in \mathbb{N}$, $\exists k \in \mathbb{N}, n = 5k+2 \Rightarrow \exists j \in \mathbb{N}, n^2 = 5j+4$
		\end {description}
\item Proving the converse of 2.1 is false.
\item The converse of 2.1 would be $\forall n \in \mathbb{N}$, $\exists j \in \mathbb{N}, n^2 = 5j+4 \Rightarrow \exists k \in \mathbb{N}, n = 5k+2$
\item The negation of the converse of 2.1 would be $\exists n \in \mathbb{N}$, $\exists j \in \mathbb{N}, n^2 = 5j+4 \wedge \forall k \in \mathbb{N}, n \neq 5k+2$
		\begin {description}
				\item Choose $n = 9$ \# $n \in \mathbb{N}$
				\item Then, $n^2$ can be written in the form of $5j + 4$, where $j = 1$. \# Then, $j \in \mathbb{N}$
				\item Then, $n = 3$ \# Simply taking the square root of $n^2$
				\item But $n$ cannot be written in the form of $5k + 3$, for any $k$.
				\item Thus, $\exists j \in \mathbb{N}, n^2 = 5j+4 \wedge\forall k \in \mathbb{N}, n \neq 5k+2$
		\end{description}
		\item Then, $\exists n \in \mathbb{N}$, $\exists j \in \mathbb{N}, n^2 = 5j+4 \wedge \forall k \in \mathbb{N}, n \neq 5k+2$ \# We've proved the negation of the converse, so we disproved the converse.
\item Proving claim 2.2 is true, and disproving its converse.
		\begin{description}
				\item Assume $m, n \in \mathbb{R}$ \# generic
						\begin{description}
						\item Let $k_0$, $j_0$ be a natural numbers
						\item Assume $m = 7k_0 + 3$ and $n = 7j_0 + 4$ \# Antecedent
								\begin{description}
								\item Then, evaluating $mn$, we get $mn = (7k_0 + 3)(7j_0 +4)$
								\item $mn = 49k_0j_0 + 28k_0 + 21j_0 + 12$ \# Expand the product
								\item $mn = 49k_0j_0 + 28k_0 + 21j_0 + 7 + 5$ \# Change 12 into 7+5
								\item $mn = 7(7k_0j_0 + 6k_0 + 3j_0 + 1) +5$ \# Factor out 7 from the first 4 terms.
								\end {description}

						\item Let $i_0 = 7k_0j_0 + 6k_0 + 3j_0 + 1$
						\item Then $i_0 \in \mathbb{N}$ \# Natural numbers are closed under addition and multiplication.
						\item Then $mn = 7i_0 + 5$ \# Substitute in i
						\end{description}
				\item Then, $\forall m,n \in \mathbb{N}$, ($\exists k \in \mathbb{N}, m = 7k + 3$) $\wedge$ ($\exists j \in \mathbb{N}, n = 7j + 4$) $\Rightarrow$ ($\exists i \in \mathbb{N}, mn = 7i + 5$)
		\end{description}
\item Proving the converse of 2.2 is false. We will continue by finding the converse, and then proving the negation of its contrapositive.
\begin{description}
		\item The converse of 2.2 is $\forall m,n$ ($\exists i \in \mathbb{N}, mn = 7i + 5$) $\Rightarrow$ ($\exists k \in \mathbb{N}, m = 7k + 3$) $\wedge$ ($\exists j \in \mathbb{N}, n = 7j + 4$)
		\item Its contrapositive is $\forall m,n$ ($\forall k \in \mathbb{N}, m \neq 7k+3$) $\vee$ ($\forall j \in \mathbb{N}, n \neq 7j + 4$) $\Rightarrow$ $\forall i \in \mathbb{N} mn \neq 7i + 5$
		\item The negation is $\exists m,n$ ($\forall k \in \mathbb{N}, m \neq 7k + 3$) $\vee$ ($\forall j \in \mathbb{N}, n \neq 7j + 4$) $\Rightarrow$ ($\exists i \in \mathbb{N}, mn = 7i + 5)$.
		\begin{description}
			\item Let $m \in \mathbb{N}, m = 10$
			\item Then $\forall k \in \mathbb{N}, m \neq 7k + 3$ is false, there exists $k$ with value of 1 which causes it to evaluate false.
			\item Let $n \in \mathbb{N}, n = 11$
			\item Then $\forall j \in \mathbb{N}, n \neq 7j + 4$ is false, there exists $j$ with value of 1 which causes it to evaluate false.
			\item Then the antecedent is always false and, therefore, the implication is true.
		\end{description}
		\item Then, $\exists m,n$ ($\forall k \in \mathbb{N}, m \neq 7k + 3$) $\vee$ ($\forall j in \mathbb{N}, n \neq 7j + 4$) $\Rightarrow$ $\exists i \in \mathbb{N} mn = 7i + 5$ \# Assumed antecendent, got consequent.
\end{description}
\end{enumerate}

\end{document}

% LocalWords:  brachial
