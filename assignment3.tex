\documentclass{article}
\usepackage{amsmath,amssymb}
\usepackage{hyperref}
\title{CSC165 Fall 2014, Assignment 3}
\author{Anish Krishna, Connor Peet, George Wu}
\renewcommand{\today}{~}
\hypersetup{pdfpagemode=Fullscreen,
  colorlinks=true,
  linkfileprefix={}}
\newcommand{\floor}[1]{\lfloor #1\rfloor}
\begin{document}
\maketitle


\begin{enumerate}
\item Proving the statement 1 true.
    \begin{description}
    \item Assume $e \in \mathbb{R}^+$ \# generic
    		\begin{description}
    		\item Let $d = e$
    		\item Then $d \in \mathbb{R}^+$ \# e is a positive, real number.
    		\item Assume $x, y \in \mathbb{R}^+$ and that $|x - y| > d$ \# Generic, antecedent assumption
    			\begin{description}
    			\item Then $x + y = |x + y|$ \# $\mathbb{R}^+$ closed by addition, the absolute value of a positive number is that number.
    			\item Also $x + y > |x - y|$ \# The sum of two positive real numbers will always be greater than their difference.
    			\item Then $|x + y| > d$ \# Sub in, by antecedent
    			\item Then $|x + y| > e$
    			\end{description}
    		\end{description}
    \item Then $\forall e \in \mathbb{R}^+, \exists d \in \mathbb{R}^+, \forall x, y \in \mathbb{R}^+, |x - y| > d \implies |x + y| > e$.
    \end{description}
\item Proving statement 2 true, where $f(n) = 6n^3 - 4n^2 + 3n + 2$ and $g(n) = 5n^3 - n^2 + n + 1$
	\begin{description}
	\item Let $c = 1$
	\item Then $c \in \mathbb{R}^+$ \# 1 is a positive real number.
	\item Let $b = 3$
	\item Then $c \in \mathbb{N}$ \# 3 is a natural number.
	\item Assume $n \in \mathbb{N}$
		\begin{description}
		\item Then $f(n) = 6n^3 - 4n^2 + 3n + 2$
		\item Then $f(n) \geq 6n^3 - 4n^2 + n + 1$ \# Subtract $2n + 1$, which is greater than zero because $n \in \mathbb{R}^+$
		\item Then $f(n) \geq 5n^3 - n^2 + n + 1$ \# Subtract $n^3 - 3n^2$, which is greater than or equal to zero because $n \geq 3$
		\item Then $f(n) \geq g(n)$ \# Substitue in $g(n)$
		\item Then $f(n) \geq cg(n)$ \# $c = 1$, identity property of multiplication
		\end{description}
	\item Then $\exists c \in \mathbb{R}^+, \exists B \in \mathbb{N}, \forall n \in n, n \geq B \implies f(n) \geq cg(n)$
	\item Then $6n^3 - 4n^2 + 3n + 2 \in \Omega(5n^3 - n^2 + n + 1)$
	\end{description}
	
\item Proving number 3 is false.
	\begin{description}
		\item Using l'Hopital's rule, we know that:	
		\begin{equation*}
		\lim_{x\to\infty}\frac{15n^2}{3 \times 2^n}=0
		\end{equation*}
		\item Negation of the given statement is the following: $\forall B \in \mathbb{N}, \forall c \in \mathbb{R}, \exists n \in \mathbb{N}, n\geq B \wedge 15n^2 < c\times 3 \times 2^n$
		\item Assume $B \in \mathbb{N}, c \in \mathbb{R}^+$.
		\begin{description}
			\item Then $\forall e \in \mathbb{R}^+, \exists M \in \mathbb{N}, \forall n \in \mathbb{N}, n \geq M \Rightarrow L - e < \frac{15n^2}{3\times n^2}< L + e$ \# Definition of limit
			\item Then $\forall e \in \mathbb{R}^+, \exists M \in \mathbb{N}, \forall n \in \mathbb{N}, n \geq M \Rightarrow -e < \frac{15n^2}{3\times n^2}< e$ \# L = 0, by l'Hopital's rule
			\item Then, $\exists M \in \mathbb{N}, \forall n \in \mathbb{N}, n \geq M \Rightarrow \frac{15n^2}{3\times n^2} < c$ \# We let e = c.
			\item Let $M$ be such that $\forall n \in \mathbb{N}, n \geq M \Rightarrow \frac{15n^2}{3\times n^2} < c$ and choose $n = \max(M,B)$
			\item Then $n \in \mathbb{N}$
			\item Then $n \geq B$ \# By definition of maximum.
			\item Then $15n^2 < c \times 3 \times 2^n$\#Multiplying $3\times 2^n$ over the inequality. We know this is true because $2^n$ is positive, so this will not change the inequality.
			\item Then $n \geq B$ and $15n^2 < c\times 3\times 2^n$.
		\end{description}
		\item Then $\forall B \in \mathbb{N}, \forall c \in \mathbb{R}, \exists n \in \mathbb{N}, n\geq B \wedge 15n^2 < c\times 3 \times 2^n$
	\end{description}
	
\item Proving number 4 true.
\begin{description}
	\item We know that:
	\begin{equation*}
		\lim_{x\to\infty}\frac{2^n}{3^n}=\lim_{x\to\infty} \left(\frac{2}{3}\right)^n = 0
	\end{equation*}
	\item By the definition of the limit, then $\forall e \in \mathbb{R}^+, \exists M \in \mathbb{N}, \forall n \in \mathbb{N}, n \geq M \Rightarrow -e< \frac{2^{n}}{3^{n}} < e$
	\item Then, we know that there is some $e'$ that satisfies this definition, which we will set to $1$.
	\item Pick $B = M$, such that the definition holds. \# So $B \in \mathbb{N}$
	\item Pick $c = e' = 1$, such that the definition holds. \# So $c \in \mathbb{R}^+$
	\item Assume $n$ to be any natural number and $n \geq B$
	\begin{description}
		\item Then $-c<\frac{2^n}{3^n}<c$ \# Since we know that $n\geq B$ and by extension $n\geq M$, satisifying the limit definition.
		\item Then $\frac{2^n}{3^n} < 1$ \# c= 1, and dismissing the negative side, since the fraction will always remain positive.
		\item Then $2^n < 3^n$ \# since $n \geq B$, so we know that $3^n$ is positive and will not change the inequality sign.
		\item Then $n \geq B \Rightarrow n\geq M \Rightarrow 2^n < 3^n$ \# Assumed antecedent and got conclusion.
	\end{description}
	\item Then $\exists B \in \mathbb{N}, \exists c \in \mathbb{R}^+, \forall n \in \mathbb{N}, n \geq B \Rightarrow 2^n \leq 3^n$
\end {description}
\item Proving number 5 is false.
\begin{description}
	\item The negation of the statement would be: $\exists f,g \in \mathbb{F}, f\notin O(g) \wedge f \notin \Omega (g)$
	\item Choose $g(n) = n^2$
	\item Choose
	\begin {displaymath}
		f(x)= \left\{
			\begin{array}{lr}
				n   &: n\  even\\
				n^3 &: n\  odd
			\end{array}
		\right.
	\end {displaymath}
	\item Now we prove $\forall B \in \mathbb{N}, \forall c \in \mathbb{R}^+, \exists N \in \mathbb{N}, N >= B \wedge f(x) > g(x)$
	\item Let $B$ be any natural number. Then $B \in \mathbb{N}$.
	\item Let $c$ be any positive real. Then $c \in \mathbb{R}^+$.
	\item Case 1: B is even.
	\begin {description}
		\item Choose $N = \max(B+1,\left\lceil c + 1\right\rceil)$ \# Then $N \in \mathbb{N}$, since $B+1 \in \mathbb{N}$ and $\left\lceil c + 1\right\rceil \in \mathbb{N}$
		\item Then $N \geq B$ \#By definition of max.
		\item Then $N = \max(B+1,\left\lceil c + 1\right\rceil) \geq \left\lceil c + 1\right\rceil \geq c + 1 \geq c$
		\item Then $N^3 = N \times N^2 \geq c\times N^2$ \# Since $N \geq c$, and $N$ is odd. 
	\end {description}
	\item Case 2: B is odd
	\begin {description}
		\item Choose $N = \max(B,\left\lceil c + 1\right\rceil)$ \# Then $N \in \mathbb{N}$, since $B \in \mathbb{N}$ and $\left\lceil c + 1\right\rceil \in \mathbb{N}$
		\item Then $N \geq B$ \# By definition of max.
		\item Then $N = \max(B,\left\lceil c + 1\right\rceil) \geq \left\lceil c + 1\right\rceil \geq c+1 \geq c$
		\item Then $N^3 = N \times N^2 \geq c\times N^2$ \# Since $N \geq c$, and $N$ is odd.
	\end {description}
	\item Then $x^3 > cx^2$
	\item Then $f \notin O(g)$
	
	\item Now, to prove that $\forall B \in \mathbb{N}, \forall c \in \mathbb{R}^+, \exists N \in \mathbb{N}, N >= B \wedge f(N) < g(N)$
	\item Case 1: B is even.
	\begin{description}
		\item Choose $N = \max(B,\left\lceil c - 1\right\rceil)$ \# Then $N \in \mathbb{N}$, since $B \in \mathbb{N}$ and $\left\lceil c - 1\right\rceil \in \mathbb{N}$
		\item Then $N \geq B$ \# By definition of max.
		\item Then $N \leq c$ \# Since $n$ can be at most $\left\lceil c - 1\right\rceil$
		\item Then $N < N^3 = N \times N^2 < c\times N^2$ \# Since $N \leq c$, and $N$ is even.
	\end {description}
	\item Case 2: B is odd.
	\begin{description}
		\item Choose $N = \max(B+1,\left\lceil c - 1\right\rceil)$ \# Then $N \in \mathbb{N}$, since $B+1 \in \mathbb{N}$ and $\left\lceil c - 1\right\rceil \in \mathbb{N}$
		\item Then $N \geq B$ \# By definition of max.
		\item Then $N \leq c$ \# Since $n$ can be at most $\left\lceil c - 1\right\rceil$
		\item Then $N < N^3 = N \times N^2 < c\times N^2$ \# Since $N \leq c$, and $N$ is even.
	\end {description}
	\item Then $N^3 < cN^2$
	\item Then $f \notin \Omega(g)$
	\item Then, $\exists f, \exists g \in \mathbb{F}, f\notin O(g) \wedge f\notin \Omega(g)$
\end{description}

\item Proving that the function \verb+meaning_of_life+ below is not computable:   %6
Using proof by contradiction to prove non-computability. 
\begin{description}
	\item Assume  \verb+meaning_of_life+ is computable 
	\begin{description}
		\item Then 
		\begin{verbatim}
		def meaning_of_life(f, I):
		 	"""Return True if f(I) returns 42, False otherwise."""
			# Implementation omitted

		def halt(f, i) :
		   def P(x) :
		      f(i)
		      if True:
		         return 42
		   return meaning_of_life(P, x)
		\end{verbatim}
		\item Then the if statement block under halt function executes iff f(i) halts.
		\item Then halt(f, i) returns True if f(i) halts, else it will return False. 
		\item Contradiction. Halt functions are non-computable  \# General specification of halt functionality
	\end{description}
	\item Then  \verb+meaning_of_life+ is non-computable 


\end {description}
\end{enumerate}

\end{document}

% LocalWords:  brachial
