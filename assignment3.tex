\documentclass{article}
\usepackage{amsmath,amssymb}
\usepackage{hyperref}
\title{CSC165 Fall 2014, Assignment 3}
\author{Anish Krishna, Connor Peet, George Wu}
\renewcommand{\today}{~}
\hypersetup{pdfpagemode=Fullscreen,
  colorlinks=true,
  linkfileprefix={}}
\newcommand{\floor}[1]{\lfloor #1\rfloor}
\begin{document}
\maketitle


\begin{enumerate}
\item Proving the statement 1 true.
    \begin{description}
    \item Assume $e \in \mathbb{R}^+$ \# generic
    		\begin{description}
    		\item Let $d = e$
    		\item Then $d \in \mathbb{R}^+$ \# e is a positive, real number.
    		\item Assume $x, y \in \mathbb{R}^+$ and that $|x - y| > d$ \# Generic, antecedent assumption
    			\begin{description}
    			\item Then $x + y = |x + y|$ \# $\mathbb{R}^+$ closed by addition, the absolute value of a positive number is that number.
    			\item Also $x + y > |x - y|$ \# The sum of two positive real numbers will always be greater than their difference.
    			\item Then $|x + y| > d$ \# Sub in, by antecedent
    			\item Then $|x + y| > e$
    			\end{description}
    		\end{description}
    \item Then $\forall e \in \mathbb{R}^+, \exists d \in \mathbb{R}^+, \forall x, y \in \mathbb{R}^+, |x - y| > d \implies |x + y| > e$.
    \end{description}
\item Proving statement 2 true, where $f(n) = 6n^3 - 4n^2 + 3n + 2$ and $g(n) = 5n^3 - n^2 + n + 1$
	\begin{description}
	\item Let $c = 1$
	\item Then $c \in \mathbb{R}^+$ \# 1 is a positive real number.
	\item Let $b = 3$
	\item Then $c \in \mathbb{N}$ \# 3 is a natural number.
	\item Assume $n \in \mathbb{N}$
		\begin{description}
		\item Then $f(n) = 6n^3 - 4n^2 + 3n + 2$
		\item Then $f(n) \geq 6n^3 - 4n^2 + n + 1$ \# Subtract $2n + 1$, which is greater than zero because $n \in \mathbb{R}^+$
		\item Then $f(n) \geq 5n^3 - n^2 + n + 1$ \# Subtract $n^3 - 3n^2$, which is greater than or equal to zero because $n \geq 3$
		\item Then $f(n) \geq g(n)$ \# Substitue in $g(n)$
		\item Then $f(n) \geq cg(n)$ \# $c = 1$, identity property of multiplication
		\end{description}
	\item Then $\exists c \in \mathbb{R}^+, \exists B \in \mathbb{N}, \forall n \in n, n \geq B \implies f(n) \geq cg(n)$
	\item Then $6n^3 - 4n^2 + 3n + 2 \in \Omega(5n^3 - n^2 + n + 1)$
	\end{description}
	
\item Proving number 3 is false.
	\begin{description}
		\item Using l'Hopital's rule, we know that:	
		\begin{equation*}
		\lim_{x\to\infty}\frac{15n^2}{3 \times 2^n}=0
		\end{equation*}
		\item Negation of the given statement is the following: $\forall B \in \mathbb{N}, \forall c \in \mathbb{R}, \exists n \in \mathbb{N}, n\geq B \wedge 15n^2 < c\times 3 \times 2^n$
		\item Assume $B \in \mathbb{N}, c \in \mathbb{R}^+$.
		\begin{description}
			\item Then $\forall e \in \mathbb{R}^+, \exists M \in \mathbb{N}, \forall n \in \mathbb{N}, n \geq M \Rightarrow L - e < \frac{15n^2}{3\times n^2}< L + e$ \# Definition of limit
			\item Then $\forall e \in \mathbb{R}^+, \exists M \in \mathbb{N}, \forall n \in \mathbb{N}, n \geq M \Rightarrow -e < \frac{15n^2}{3\times n^2}< e$ \# L = 0, by l'Hopital's rule
			\item Then, $\exists M \in \mathbb{N}, \forall n \in \mathbb{N}, n \geq M \Rightarrow \frac{15n^2}{3\times n^2} < c$ \# We let e = c.
			\item Let $M$ be such that $\forall n \in \mathbb{N}, n \geq M \Rightarrow \frac{15n^2}{3\times n^2} < c$ and choose $n = \max(M,B)$
			\item Then $n \in \mathbb{N}$
			\item Then $n \geq B$ \# By definition of maximum.
			\item Then $15n^2 < c \times 3 \times 2^n$\#Multiplying $3\times 2^n$ over the inequality. We know this is true because $2^n$ is positive, so this will not change the inequality.
			\item Then $n \geq B$ and $15n^2 < c\times 3\times 2^n$.
		\end{description}
		\item Then $\forall B \in \mathbb{N}, \forall c \in \mathbb{R}, \exists n \in \mathbb{N}, n\geq B \wedge 15n^2 < c\times 3 \times 2^n$
	\end{description}
	
\item Proving number 4 true.
\begin{description}
	\item We know that:
	\begin{equation*}
		\lim_{x\to\infty}\frac{2^n}{3^n}=\lim_{x\to\infty} \left(\frac{2}{3}\right)^n = 0
	\end{equation*}
	\item Pick $c = 1$, $B = 0$
	\item Assume $n$ to be any natural number and $n \geq B$
	\begin{description}
		\item Then $\exists M \in \mathbb{N}, \forall n_1 \in \mathbb{N}, n_1 \geq M \Rightarrow -c< \frac{2^{n_1}}{3^{n_1}} < c$ \# We know that the limit exists, so we simply set e = c.
		\item Choose $M = B$.
		\item Then we know that $\forall n_1 \in \mathbb{N}, n_1 \geq B \Rightarrow \frac{2^{n_1}}{3^{n_1}} < 1$ \# Since c = 1.
		\item Then $n_1 \in \mathbb{N}$ and $n_1 \geq M = B$
		\item Then $n_1$ describes the same set of numbers as $n$
		\item Then $\frac{2^{n_1}}{3^{n_1}} = \frac{2^n}{3^n} < 1$ \# Setting c = 1.
		\item Then $2^n < 3^n$ \# since $n \geq M$, so we know that $3^n$ is positive.
		\item Then $n \geq B \Rightarrow 2^n < 3^n$ \# Assumed antecedent and got conclusion.
	\end{description}
	\item Then $\exists B \in \mathbb{N}, \exists c \in \mathbb{R}^+, \forall n \in \mathbb{N}, n \geq B \Rightarrow 2^n \leq 3^n$
\end {description}
\end{enumerate}

\end{document}

% LocalWords:  brachial
