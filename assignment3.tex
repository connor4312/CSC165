\documentclass{article}
\usepackage{amsmath,amssymb}
\usepackage{hyperref}
\title{CSC165 Fall 2014, Assignment 3}
\author{Anish Krishna, Connor Peet, George Wu}
\renewcommand{\today}{~}
\hypersetup{pdfpagemode=Fullscreen,
  colorlinks=true,
  linkfileprefix={}}
\newcommand{\floor}[1]{\lfloor #1\rfloor}
\begin{document}
\maketitle


\begin{enumerate}
\item Proving the statement 1 true. Recall the definition of floor:
    \begin{equation*}
        \floor{x} \in\mathbb{Z} \wedge \floor{x}
        \leq x \wedge (\forall z \in\mathbb{Z}, z \leq x \Rightarrow z
        \leq \floor{x})
    \end{equation*}
    Then:
    \begin{description}
    \item Assume $e \in \mathbb{R}^+$ \# generic
    		\begin{description}
    		\item Let $d = e + 1$
    		\item Then $d \in \mathbb{R}^+$ \# $\mathbb{R}^+$ bounded by addition
    		\item Assume $x, y \in \mathbb{R}^+$ and that $|x - y| > d$ \# Generic, antecedent assumption
    			\begin{description}
    			\item Then $\floor{x - y} > e + 1$ \# Antecedent
    			\item Also $x + y > x - y$ \# Since $x, y \in \mathbb{R}^+$
    			\item Then $\floor{x + y} \geq \floor{x - y}$ \# By definition the floor of a larger number will never be less than the floor of a smaller number: $\floor{x} \leq x \wedge (\forall z \in\mathbb{Z}, z \leq x \Rightarrow z \leq \floor{x})$
    			\item Then $\floor{x + y} \geq e + 1 > e$ \# Sub in with antecedent
    			\item Then $\floor{x + y} > e$
    			\end{description}
    		\end{description}
    \item Then $\forall e \in \mathbb{R}^+, \exists d \in \mathbb{R}^+, \forall x, y \in \mathbb{R}^+, |x - y| > d \implies |x + y| > e$.
    \end{description}
\item Proving statement 2 true, where $f(n) = 6n^3 - 4n^2 + 3n + 2$ and $g(n) = 5n^3 - n^2 + n + 1$
	\begin{description}
	\item Let $c = 1$
	\item Then $c \in \mathbb{R}^+$ \# 1 is a positive real number.
	\item Let $b = 3$
	\item Then $c \in \mathbb{N}$ \# 3 is a natural number.
	\item Assume $n \in \mathbb{N}$
		\begin{description}
		\item Then $f(n) = 6n^3 - 4n^2 + 3n + 2$
		\item Then $f(n) \geq 6n^3 - 4n^2 + n + 1$ \# Subtract $2n + 1$, which is greater than zero because $n \in \mathbb{R}^+$
		\item Then $f(n) \geq 5n^3 - n^2 + n + 1$ \# Subtract $n^3 - 3n^2$, which is greater than or equal to zero because $n \geq 3$
		\item Then $f(n) \geq g(n)$ \# Substitue in $g(n)$
		\item Then $f(n) \geq cg(n)$ \# $c = 1$, identity property of multiplication
		\end{description}
	\item Then $\exists c \in \mathbb{R}^+, \exists B \in \mathbb{N}, \forall n \in n, n \geq B \implies f(n) \geq cg(n)$
	\item Then $6n^3 - 4n^2 + 3n + 2 \in \Omega(5n^3 - n^2 + n + 1)$
	\end{description}
\end{enumerate}

\end{document}

% LocalWords:  brachial
